%%%%%%%%%%%%%%%%%%%%%%%%%%%%%%%%%%%%%%%%%%%%%%%%%%%%%%%%%%
%                                                                                      %
%         Univesity College London Project LaTex Template            %
%                                                                                      %
%%%%%%%%%%%%%%%%%%%%%%%%%%%%%%%%%%%%%%%%%%%%%%%%%%%%%%%%%%
%
%   Author: Alex Charles           Email: aep.charles@gmail.com
%
% -----------------------------------------------------------------------------------
%      PACKAGES & OTHER DOCUMENT CONFIGURATIONS
% -----------------------------------------------------------------------------------
\documentclass[fontsize=11pt]{extarticle}

\usepackage[utf8]{inputenc}
\usepackage[T1]{fontenc}
\usepackage[british]{babel}
% ----------NEW BIBLATEX BIBLIOGRAPHY-----------------------------------------------
\usepackage[eprint=false,backend=bibtex,style = ieee]{biblatex} % Upgrades Bibliography Block Ragged helps break lines in url fixes error

\addbibresource{BibFile.bib} %%% For biblatex
%e.g to add page number \footfullcite[chapter, p.~215]{AAIB}
% of \footnote{Footnote text goes here}
% This allows can use footfullcite commands
% Note urldate field must be in yyyy-mm-dd to work - use online type
% Remeber to use \printbibliography in the footer
% -----------------------------------------------------------------------------------
% \usepackage{sectsty}
\usepackage{url}

%%% --- The following two lines are what needs to be added --- %%%
\setcounter{biburllcpenalty}{7000}
\setcounter{biburlucpenalty}{8000}

\usepackage{amssymb,amsmath}
\numberwithin{figure}{section} %%%%% <<<<<< Puts Figure Numbering into Sections
\usepackage{ifxetex,ifluatex}  %<<<<<<<<< Edit FONT HERE
\ifnum 0\ifxetex 1\fi\ifluatex 1\fi=0 % if pdftex
  \usepackage[T1]{fontenc}
  \usepackage[utf8]{inputenc}
\else % if luatex or xelatex
  \ifxetex
    \usepackage{mathspec}
    \setmainfont[
 BoldFont={AvenirNext-Medium},ItalicFont={AvenirNext-Italic},
 BoldItalicFont={AvenirNext-MediumItalic}]{AvenirNext-Regular}
  \else
  % Font Package for XeLatex
    \usepackage{fontspec}
    \setmainfont{AvenirNext-Regular}
  \fi
  \defaultfontfeatures{Ligatures=TeX,Scale=MatchLowercase}
\fi
\usepackage[fit]{truncate} %Truncates headers that are too long
\usepackage[headheight=26pt,headsep=0.15cm]{geometry}
\usepackage{fancyhdr}
\usepackage{lastpage}
\usepackage{extramarks}
\usepackage{gensymb}
\usepackage{lipsum}
\usepackage{float}
\usepackage{graphicx}
\graphicspath{{TempImg/}{Img/}}%<<<<<<<<< Location of Template Images and Other Images, Add folders here
\usepackage{subfig}
\usepackage{wrapfig}
\usepackage[font ={small,it}]{caption}
\usepackage{amsfonts,amsthm} % Math packages
% \usepackage{cite}
\usepackage{csquotes}
%    \MakeAutoQuote{‘}{’}
%    \MakeAutoQuote*{“}{”} %corrects quote marks
\usepackage{enumitem} % resume numbered lists
\usepackage{multicol} %for mulitple colums in lists
\usepackage{booktabs} %<<<<<<<<< Table drawing package
\usepackage[table,xcdraw]{xcolor} %<<<<<<<<< Table drawing package
\usepackage{svg}
\usepackage{scrextend} %call footnotes
\usepackage[colorlinks, linkcolor = black, citecolor = black, filecolor = black, urlcolor = blue]{hyperref} % Creates Hyperlinks for references - add [colorlinks] for coloured hyperlinks
\usepackage{changepage} %Allows Adjust width to be used for the document (indenting paragraphs)
\usepackage{pdfpages} %Allows Pdfpages to be added to the document use \includepdf[pages={1}]{myfile.pdf}
\usepackage{pdflscape} %Change Pages from Portrait to Landscape
\usepackage{color,soul} %% Highlights text for markup
% \usepackage[compact]{titlesec}

\usepackage{pdfpages} %Allows Pdfpages to be added to the document use \includepdf[pages={1}]{myfile.pdf}
%%%%%%%For Condensed Report Format%%%%%%%%%%%%%%
\usepackage{titlesec}
\titlespacing\section{0pt}{6pt plus 2pt minus 2pt}{0pt plus 2pt minus 2pt}
\titlespacing\subsection{0pt}{0pt plus 3pt minus 2pt}{-3pt plus 2pt minus 2pt}
\titlespacing\subsubsection{0pt}{0pt plus 2pt minus 2pt}{-6pt plus 2pt minus 2pt}
\titlespacing\subsubsubsection{0pt}{-6pt plus 2pt minus 2pt}{-6pt plus 2pt minus 2pt}
\setlength{\multicolsep}{-1pt plus 2.0pt minus 1.5pt}% 50% of original values

% \titlespacing*{\section}{0pt}{1.1\baselineskip}{\baselineskip}

\renewcommand*{\thefootnote}{\alph{footnote}} %%% Changes footnotes to letters
\usepackage[bottom]{footmisc} %%% Pushes footnote to bottom and to the margin

\DeclareCiteCommand{\footcite}[\mkbibfootnote]
{\usebibmacro{cite:init}%
\usebibmacro{prenote}}
{\usebibmacro{citeindex}%
\printtext[brackets]{\usebibmacro{cite:comp}}}
{\multicitedelim}
{\usebibmacro{cite:dump}%
\usebibmacro{postnote}}

\newenvironment{indentpara}{\begin{adjustwidth}{2cm}{}}{\end{adjustwidth}} %Declare adjust width wiht indentpara
\renewcommand{\labelitemii}{$\circ$}
\renewcommand{\labelitemiii}{$\diamond$}
\renewcommand{\labelitemiii}{$\cdot$}

% -----------------------------------------------------------------------------------
%                 Code
% -----------------------------------------------------------------------------------
\usepackage{listings}
\lstset{inputpath=Code/}
\usepackage{color}
\definecolor{mygreen}{RGB}{28,172,0} % color values Red, Green, Blue
\definecolor{mylilas}{RGB}{170,55,241}

\lstset{language=Matlab,%
    %basicstyle=\color{red},
    breaklines=true,%
    basicstyle=\small,
    morekeywords={matlab2tikz},
    keywordstyle=\color{blue},%
    morekeywords=[2]{1}, keywordstyle=[2]{\color{black}},
    identifierstyle=\color{black},%
    stringstyle=\color{mylilas},
    commentstyle=\color{mygreen},%
    showstringspaces=false,%without this there will be a symbol in the places where there is a space
    numbers=left,%
    numberstyle={\tiny \color{black}},% size of the numbers
    numbersep=9pt, % this defines how far the numbers are from the text
    emph=[1]{for,end,break},emphstyle=[1]\color{red}, %some words to emphasise
    %emph=[2]{word1,word2}, emphstyle=[2]{style},
}

%% To Add Code Use :
% \lstinputlisting{myfun.m}
%% To input a file or :
% \begin{figure}[h]
% \begin{lstlisting}[language=Matlab]
% \end{lstlisting}
% \catpion{code}
% \end{figure}


% -----------------------------------------------------------------------------------
%                 Quotes
% -----------------------------------------------------------------------------------

\usepackage{epigraph}
% \epigraphsize{\small}% Default
\setlength\epigraphwidth{12cm}
\setlength\epigraphrule{0pt}

\usepackage{etoolbox}
\apptocmd{\sloppy}{\hbadness 10000\relax}{}{}%%%% > Removes Url bibliography warnings
\makeatletter
\patchcmd{\epigraph}{\@epitext{#1}}{\itshape\@epitext{#1}}{}{}
\makeatother

%%%% > For Quotes Use \epigraph{"Quote"}{ - \textup{Author}, Book}

% -----------------------------------------------------------------------------------
%                   NAMES & CLASS DEFINITION %<<<<<<<<< INSERT DETAILS HERE
% -----------------------------------------------------------------------------------
\newcommand{\AssignmentTitle}{IXN Website - Design and Implementation}
\newcommand{\ModuleTitle}{compgc02 App Design}
\newcommand{\DegreeTitle}{MSc Computer Science}
\newcommand{\University}{University College London}
\newcommand{\Faculty}{Faculty of Engineering Sciences}
\newcommand{\UniCrest}{logoucl.png}
\newcommand{\UniLogo}{logoucl.png}%<<<<<<<<< Make Sure Files are in the Template
\newcommand{\GroupName}{group 3 - ixn website}
\newcommand{\StudentNameA}{Alexander Charles}
\newcommand{\StudentNumberA}{alexander.charles.17@ucl.ac.uk}
\newcommand{\StudentNameC}{Giovanni Tenderini}
\newcommand{\StudentNumberC}{giovanni.tenderini.17@ucl.ac.uk}
\newcommand{\StudentNameB}{Pheobe Staab}
\newcommand{\StudentNumberB}{phoebe.staab.17@ucl.ac.uk}
\newcommand{\SupervisorNameA}{Yun Fu}
\newcommand{\SupervisorEmailA}{yun.fu@ucl.ac.uk}


% -----------------------------------------------------------------------------------
%        PACKAGES FOR MARKDOWN CONVERSION - FOR USE If Using Markdown to Latex
% -----------------------------------------------------------------------------------
\usepackage{fixltx2e} % provides \textsubscript
% use upquote if available, for straight quotes in verbatim environments
\IfFileExists{upquote.sty}{\usepackage{upquote}}{}
% use microtype if available
\IfFileExists{microtype.sty}{%
\usepackage{microtype}
\UseMicrotypeSet[protrusion]{basicmath} % disable protrusion for tt fonts
}{}
\hypersetup{unicode=true,
            pdftitle={\AssignmentTitle},
            pdfauthor={\StudentNameA},
            pdfborder={0 0 0},
            breaklinks=true}
\urlstyle{same}  % don't use monospace font for urls
\usepackage{fancyvrb}
\VerbatimFootnotes % allows verbatim text in footnotes
\usepackage{longtable,booktabs}
\IfFileExists{parskip.sty}{%
\usepackage{parskip}
}{% else
\setlength{\parindent}{0pt}s
\setlength{\parskip}{6pt plus 2pt minus 1pt}
}
\setlength{\emergencystretch}{3em}  % prevent overfull lines
\providecommand{\tightlist}{%
  \setlength{\itemsep}{0pt}\setlength{\parskip}{0pt}}
% \setcounter{secnumdepth}{0}
% Redefines (sub)paragraphs to behave more like sections
\ifx\paragraph\undefined\else
\let\oldparagraph\paragraph
\renewcommand{\paragraph}[1]{\oldparagraph{#1}\mbox{}}
\fi
\ifx\subparagraph\undefined\else
\let\oldsubparagraph\subparagraph
\renewcommand{\subparagraph}[1]{\oldsubparagraph{#1}\mbox{}}
\fi

% -----------------------------------------------------------------------------------
%                   WORD COUTNER - for XeLaTex
% -----------------------------------------------------------------------------------
% \usepackage{xesearch}
% \newcounter{words}
% \newenvironment{counted}{%
%   \setcounter{words}{0}
%   \SearchList!{wordcount}{\stepcounter{words}}
%     {a?,b?,c?,d?,e?,f?,g?,h?,i?,j?,k?,l?,m?,
%     n?,o?,p?,q?,r?,s?,t?,u?,v?,w?,x?,y?,z?}
%   \UndoBoundary{'}
%   \SearchOrder{p;}}{%
%   \StopSearching}

% -----------------------------------------------------------------------------------
%                   MARGINS, HEADERS & FOOTERS
% -----------------------------------------------------------------------------------
 \geometry{
 left=25.4mm,
 right=25.4mm,
 top=25.4mm,
 bottom=25.4mm,
 }
\linespread{1.5}

\pagestyle{fancy}
\lhead{\includegraphics[width = 0.15\textwidth]{\UniLogo}}
% \chead{\AssignmentTitle}
% \rhead{}
\lfoot{\StudentNameA, \StudentNameB, \StudentNameC}
\cfoot{}
\rfoot{Page \thepage} %%%% note the footer is swapped when page numbering style changes
\renewcommand\headrulewidth{0.4pt}
\renewcommand\footrulewidth{0.4pt}

\setlength\parindent{0pt}
 % \setlength{\headheight}{5mm}
\newcommand{\horrule}[1]{\rule{\linewidth}{#1}}

% -----------------------------------------------------------------------------------
%               DOCUMENT STRUCTURE COMMANDS
% -----------------------------------------------------------------------------------
% To sort out the formatting of header and footer when a page...
% ... split occurs "within" a problem environment.
\newcommand{\enterProblemHeader}[1]{
\nobreak\extramarks{#1 (Cont.)}\nobreak
\nobreak\extramarks{#1}{}\nobreak
}
% To sort out the formatting of header and footer when a page...
% ... split occur "between" problem environments.
\newcommand{\exitProblemHeader}[1]{
\nobreak\extramarks{#1 (Cont.)}\nobreak
\nobreak\extramarks{#1}{}\nobreak
}

% -----------------------------------------------------------------------------------
\begin{document}


%For PDF intro
% \includepdf[pages={-}]{DP4front.pdf}

  \setlength{\abovedisplayskip}{-18pt}
  \setlength{\belowdisplayskip}{0pt}
  \setlength{\abovedisplayshortskip}{-18pt}
  \setlength{\belowdisplayshortskip}{0pt}

  \setlist[enumerate]{itemsep=-2mm}
  \setlist[itemize]{itemsep=-2mm}


%----------------------------------------------------------------------------------------
                                  %	TITLE PAGE FORMAT
%----------------------------------------------------------------------------------------
\pagenumbering{roman}
\begin{titlepage}

	\center % Center everything on the page
%----------------------------------------------------------------------------------------
%	HEADING SECTION
%----------------------------------------------------------------------------------------
		\usefont{OT1}{bch}{b}{n}
		\normalfont \normalsize \textsc{\University} \\ [10pt]
		\normalfont \normalsize \textsc{\Faculty} \\ [25pt]
%----------------------------------------------------------------------------------------
%	LOGO SECTION - Adds Univeristy Crest to the Report
%----------------------------------------------------------------------------------------
		\includegraphics[width = 0.6\textwidth]{\UniCrest}\\[0.5cm]
%----------------------------------------------------------------------------------------
%	HEADING SECTION
%----------------------------------------------------------------------------------------
		\normalfont \normalsize \textsc{\ModuleTitle} \\ [10pt]
    \normalfont \normalsize \textsc{\DegreeTitle} \\ [25pt]
%----------------------------------------------------------------------------------------
%	TITLE SECTION
%----------------------------------------------------------------------------------------
		\horrule{0.5pt} \\[0.4cm]
		\huge \textbf{\AssignmentTitle} \\
		\horrule{2pt} \\[0.5cm]
%----------------------------------------------------------------------------------------
%	HEADING SECTION
%----------------------------------------------------------------------------------------
\normalfont \normalsize \textsc{\GroupName} \\ [25pt]
%----------------------------------------------------------------------------------------
%	AUTHOR SECTION
%----------------------------------------------------------------------------------------
\begin{minipage}{0.4\textwidth}
  \begin{flushleft} \large
  \emph{Supervisors:}\\
  % Change Name
  \textbf{\SupervisorNameA}\\
  % \textbf{\SupervisorNameB}
  \end{flushleft}
\end{minipage}\qquad
\begin{minipage}{0.4\textwidth}
  \begin{flushright} \large
  \emph{Email:} \\
  \SupervisorEmailA \\
  % \SupervisorEmailB
  \end{flushright}
\end{minipage} \\[1cm]
\begin{minipage}{0.4\textwidth}
  \begin{flushleft} \large
  \emph{Authors:} \\
  	\textbf{\StudentNameA} \\
    \textbf{\StudentNameB} \\
    \textbf{\StudentNameC}
  \end{flushleft}
\end{minipage}\qquad
\begin{minipage}{0.4\textwidth}
  \begin{flushright} \large
     \emph{Authors Email:} \\
     (\StudentNumberA) \\
    (\StudentNumberB) \\
    (\StudentNumberC)
    \end{flushright}
\end{minipage}\\[1cm]

%----------------------------------------------------------------------------------------
%	DATE SECTION
%----------------------------------------------------------------------------------------
\textit{{\large \today}}\\[0.5cm]% Date, change the \today to a set date if you want to be precise

%----------------------------------------------------------------------------------------
%	Disclaimer
%----------------------------------------------------------------------------------------
{\small This report is submitted as part requirement for the MSc Computer Science degree at UCL. It is substantially the result of my own work except where explicitly indicated in the text. The report may be freely copied and distributed provided the source is explicitly acknowledged.}

% ---------------------------------
\vfill % Fill the rest of the page with whitespace
\end{titlepage}

% \setcounter{page}{3}

\newpage


% -----------------------------------------------------------------------------------
%                             	 ABSTRACT
% -----------------------------------------------------------------------------------

\addcontentsline{toc}{section}{Abstract}
\begin{abstract}
The Department of Computer Science at Univerisity College London (UCL) is strongly linked with industry partners all over the world, collaborating with and connecting such businesses to driven, capable student engineers. The Industry Exchange Network (IXN) is an educational methodology enabling UCL computer science students to enhance their degree training in a wide variety of real-world problem-solving projects as part of their courses. They work with a wide variety of clients ranging from small charities to large corporate partners. This work will discuss the full cycle development of a website for IXN intended display student work as well as useful information pertaining to the organisation. 

Effective and responsive UI design was essential to the site's development and as such, the team worked through an iterative design process, utilizing modern design tools. Subsequently, front-end and back-end development were done based on industry standards using a Roots development and production environment. 

The result was a WordPress website meeting all the necessary client requirements and more.  Responsive design, elegant and modern features and as well as practical functionality were achieved through a collaborative effort over a few short months. 

\end{abstract}
% -----------------------------------------------------------------------------------
%                              TABLE OF CONTENTS
% -----------------------------------------------------------------------------------

\tableofcontents
%
%
\newpage

\addcontentsline{toc}{section}{List of Tables}
\listoftables
\addcontentsline{toc}{section}{List of Figures}
\listoffigures
\addcontentsline{toc}{section}{List of Acronyms}
\section*{List of Acronyms}\label{acronyms}
\textbf{UCL}: Univeristy College London \\
\textbf{IXN}: Industry Exchange Network \\
\textbf{HCI}: Human Computer Interation \\
\textbf{UI}: User Interface \\
<<<<<<< HEAD
\textbf{MVP}: Minimum Viable Product \\
=======
\textbf{GCO2}: Module Code \\
>>>>>>> gioReport


\newpage

%% -----------------------------------------------------------------------------------
%%                          	  INTRODUCTION
%% -----------------------------------------------------------------------------------
\clearpage
\rfoot{Page \thepage\ of \pageref{LastPage}}
\pagenumbering{arabic}
% \begin{counted} %<<<<<<<<<<<<<<STARTS WORD COUNTER
